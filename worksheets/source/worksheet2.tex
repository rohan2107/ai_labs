\documentclass[a4paper,12pt]{article}
\usepackage{hyperref}
\usepackage{geometry}
\geometry{margin=1in}
\usepackage{amsmath}
\title{Worksheet 2}
\author{Conor Houghton}
\date{}

\begin{document}

\maketitle

\section{Introduction}

There are two main tasks in this worksheet; the first is fairly simple, take the matrix
\begin{equation}
  A=\begin{pmatrix}6&-2\\-2&9\end{pmatrix}
\end{equation}
and work out its eigenvectors and eigenvalues, write it in the form
$PDP^{\mathsf T}$ and check by multiplying that out that it gives back
the origin matrix.

The second is a bit harder, in the
github\footnote{\texttt{github.com/ematm0067/2025\_26/tree/master/worksheets}}
you will find a file called \texttt{ws2\_india.cvs}. This is a table
of how many people out of 10,000 speak each of the scheduled languages
of India in each of India's regions. The concept of scheduled
languages is a bit slippery, these are in some sense the `main
languages' but include Sanskrit which has almost no speakers, in
addition, some other languages have been bundled up with the scheduled
languages as if they were dialects rather than languages while other
languages are listed as `non-scheduled' languages. It is all a bit of a mess, but either way, we are
just interested in these data as example data and it is impressive these data are available at all!

The idea here is to perform PCA on these data, regarding each region
as a data point. Find the eigenvalues, plot the first two principal
components and see if you can interpret them. The interpretation will
be, admittedly, easier for some of you than others. Experiment with
using the covariance matrix or the correlation matrix. 


\end{document}
