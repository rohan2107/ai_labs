\documentclass[a4paper,12pt]{article}
\usepackage{hyperref}
\usepackage{geometry}
\geometry{margin=1in}
\usepackage{amsmath}
\title{Worksheet 2}
\author{Conor Houghton}
\date{}

\begin{document}

\maketitle

\section{Introduction}

There are two main tasks in this worksheet; the first is fairly simple, take the matrix
\begin{equation}
  A=\begin{pmatrix}6&-2\\-2&9\end{pmatrix}
\end{equation}
and work out its eigenvectors and eigenvalues, write it in the form
$PDP^{\mathsf T}$ and check by multiplying that out that it gives back
the origin matrix.

\begin{equation}
A=\begin{pmatrix}6&-2\\-2&9\end{pmatrix}.
\end{equation}

\section*{Eigenvalues}
Compute the characteristic polynomial:
\begin{equation}
\det(A-\lambda I)
=\det\begin{pmatrix}6-\lambda & -2\\ -2 & 9-\lambda\end{pmatrix}
=(6-\lambda)(9-\lambda)-(-2)(-2).
\end{equation}
So
\begin{equation}
\det(A-\lambda I)=(6-\lambda)(9-\lambda)-4
=54-15\lambda+\lambda^2-4
=\lambda^2-15\lambda+50.
\end{equation}
Solve
\begin{equation}
\lambda^2-15\lambda+50=0.
\end{equation}
The discriminant is
\begin{equation}
\Delta = 15^2-4\cdot 50=225-200=25,
\end{equation}
hence
\begin{equation}
\lambda=\frac{15\pm \sqrt{25}}{2}=\frac{15\pm 5}{2}.
\end{equation}
Therefore,
\begin{equation}
\lambda_1=10,\qquad \lambda_2=5.
\end{equation}

\section*{Eigenvectors}
\subsection*{Eigenvector for $\lambda_1=10$}
Solve $(A-10I)v=0$:
\begin{equation}
A-10I=\begin{pmatrix}6-10&-2\\-2&9-10\end{pmatrix}
=\begin{pmatrix}-4&-2\\-2&-1\end{pmatrix}.
\end{equation}
Let $v=\begin{pmatrix}x\\y\end{pmatrix}$. Then
\begin{equation}
-4x-2y=0 \quad\Longleftrightarrow\quad 2x+y=0 \quad\Longleftrightarrow\quad y=-2x.
\end{equation}
Take $x=1$, giving an eigenvector
\begin{equation}
v_1=\begin{pmatrix}1\\-2\end{pmatrix}.
\end{equation}

\subsection*{Eigenvector for $\lambda_2=5$}
Solve $(A-5I)v=0$:
\begin{equation}
A-5I=\begin{pmatrix}6-5&-2\\-2&9-5\end{pmatrix}
=\begin{pmatrix}1&-2\\-2&4\end{pmatrix}.
\end{equation}
Then
\begin{equation}
x-2y=0 \quad\Longleftrightarrow\quad x=2y.
\end{equation}
Take $y=1$, giving an eigenvector
\begin{equation}
v_2=\begin{pmatrix}2\\1\end{pmatrix}.
\end{equation}

\section*{Orthonormal eigenvectors}
Normalize $v_1$ and $v_2$:
\begin{equation}
\|v_1\|=\sqrt{1^2+(-2)^2}=\sqrt{5},\qquad
\|v_2\|=\sqrt{2^2+1^2}=\sqrt{5}.
\end{equation}
So the unit eigenvectors are
\begin{equation}
u_1=\frac{1}{\sqrt{5}}\begin{pmatrix}1\\-2\end{pmatrix},\qquad
u_2=\frac{1}{\sqrt{5}}\begin{pmatrix}2\\1\end{pmatrix}.
\end{equation}
(And indeed $u_1^\top u_2=\frac{1}{5}(1\cdot 2 + (-2)\cdot 1)=0$.)

\section*{Spectral decomposition $A = P D P^\top$}
Let
\begin{equation}
P=\begin{pmatrix} \,u_1 & u_2\, \end{pmatrix}
=\frac{1}{\sqrt{5}}
\begin{pmatrix}
1 & 2\\
-2 & 1
\end{pmatrix},
\qquad
D=\begin{pmatrix}
10 & 0\\
0 & 5
\end{pmatrix}.
\end{equation}
Then $P$ is orthogonal ($P^\top P=I$) and
\begin{equation}
A = P D P^\top.
\end{equation}

\section*{Check by multiplication}
First compute
\begin{equation}
P D
=\frac{1}{\sqrt{5}}
\begin{pmatrix}
1 & 2\\
-2 & 1
\end{pmatrix}
\begin{pmatrix}
10 & 0\\
0 & 5
\end{pmatrix}
=
\frac{1}{\sqrt{5}}
\begin{pmatrix}
10 & 10\\
-20 & 5
\end{pmatrix}.
\end{equation}
Now multiply by $P^\top$:
\begin{equation}
PDP^\top
=\frac{1}{\sqrt{5}}
\begin{pmatrix}
10 & 10\\
-20 & 5
\end{pmatrix}
\cdot
\frac{1}{\sqrt{5}}
\begin{pmatrix}
1 & -2\\
2 & 1
\end{pmatrix}
=
\frac{1}{5}
\begin{pmatrix}
10\cdot 1+10\cdot 2 & 10\cdot(-2)+10\cdot 1\\
-20\cdot 1+5\cdot 2 & -20\cdot(-2)+5\cdot 1
\end{pmatrix}.
\end{equation}
So
\begin{equation}
PDP^\top
=\frac{1}{5}
\begin{pmatrix}
30 & -10\\
-10 & 45
\end{pmatrix}
=
\begin{pmatrix}
6 & -2\\
-2 & 9
\end{pmatrix}
=A,
\end{equation}
as required.




\end{document}
